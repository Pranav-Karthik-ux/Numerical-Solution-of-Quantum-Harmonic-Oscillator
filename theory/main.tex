\documentclass[12pt]{article}

% -------------------------------------------------
% PACKAGES
% -------------------------------------------------
\usepackage{amsmath, amssymb}
\usepackage{physics}
\usepackage{graphicx}
\usepackage{bm}
\usepackage{geometry}
\usepackage{hyperref}
\usepackage{mathtools}

\geometry{a4paper, margin=1in}

\title{Numerical Solution of the Schr\"odinger Equation Using the RK4 Method}
\author{}
\date{}

\begin{document}
\maketitle

\section*{1. Introduction}

The Python code provided numerically solves the one-dimensional time-independent 
Schr\"odinger equation using the fourth-order Runge--Kutta (RK4) integration method.
The potential chosen is the harmonic oscillator, which allows us to illustrate 
how numerical wavefunctions are computed for different trial energies.

\section*{2. Schr\"odinger Equation}

The time-independent Schr\"odinger equation in one dimension is
\begin{equation}
    -\frac{1}{2}\frac{d^{2}\psi(x)}{dx^{2}}
    + V(x)\psi(x) = E\psi(x),
\end{equation}
where we have chosen units such that $\hbar = m = 1$.

Rearranging, we obtain a second-order ordinary differential equation (ODE):
\begin{equation}
    \frac{d^{2}\psi(x)}{dx^{2}} 
    = -2\left(E - V(x)\right)\psi(x).
\end{equation}

In the program, the harmonic oscillator potential is used:
\begin{equation}
    V(x) = \frac{1}{2}x^{2}.
\end{equation}

\section*{3. Reduction to a First-Order System}

To apply RK4, the second-order ODE must be rewritten as a system of two 
first-order equations. Define
\begin{equation}
    \phi(x) = \psi(x), \qquad 
    \phi'(x) = \frac{d\phi}{dx}.
\end{equation}

Then we have
\begin{align}
    \frac{d\phi}{dx} &= \phi', \\
    \frac{d\phi'}{dx} &= -2\left(E - V(x)\right)\phi .
\end{align}

These two equations form the system that RK4 evolves at each step.

\section*{4. Boundary Conditions}

The code allows two types of boundary conditions at $x=0$, corresponding 
to the parity of the wavefunction:

\begin{itemize}
    \item \textbf{Even (symmetric) solutions:}
    \[
        \phi(0) = 1, \qquad \phi'(0)=0,
    \]
    \item \textbf{Odd (antisymmetric) solutions:}
    \[
        \phi(0)=0, \qquad \phi'(0)=1.
    \]
\end{itemize}

The parameter \texttt{p} in the program selects which one is used.

\section*{5. Fourth-Order Runge--Kutta Method}

For a step size $h$, define the four RK4 increments as follows:

\[
k_1 = \phi', \qquad 
p_1 = -2\left(E-V(x)\right)\phi,
\]

\[
k_2 = \phi' + \frac{h}{2}p_1, \qquad
p_2 = -2\left(E-V(x+\tfrac{h}{2})\right)\left(\phi + \frac{h}{2}k_1\right),
\]

\[
k_3 = \phi' + \frac{h}{2}p_2, \qquad
p_3 = -2\left(E-V(x+\tfrac{h}{2})\right)\left(\phi + \frac{h}{2}k_2\right),
\]

\[
k_4 = \phi' + h p_3, \qquad
p_4 = -2\left(E-V(x+h)\right)\left(\phi + hk_3\right).
\]

The RK4 update formulas become
\begin{align}
\phi(x+h) &= \phi(x) + \frac{h}{6}\left(k_1 + 2k_2 + 2k_3 + k_4\right), \\
\phi'(x+h) &= \phi'(x) + \frac{h}{6}\left(p_1 + 2p_2 + 2p_3 + p_4\right).
\end{align}

These are exactly the recurrence relations implemented in the Python code.

\section*{6. Interpretation of Results}

The code integrates from $x=0$ to a user-defined maximum value $x_{\max}$.
For a chosen trial energy $E$, it generates the numerical wavefunction 
$\phi(x)$ and plots it. 

Only when $E$ is close to a true energy eigenvalue
\begin{equation}
    E_n = n + \frac{1}{2}, \qquad n=0,1,2,\ldots
\end{equation}
will the numerical solution remain finite. Otherwise the wavefunction diverges,
indicating that the trial energy is not an eigenvalue.

\section*{7. Summary}

This numerical approach illustrates how the Schr\"odinger equation can be solved 
by reformulating it into first-order ODEs and applying the RK4 method. The 
produced wavefunctions depend strongly on the choice of energy, enabling the 
shooting method to determine quantum eigenvalues.

\end{document}
